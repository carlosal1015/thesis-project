\section{Numerical Quadrature}

\begin{frame}
	\frametitle{
		\secname~(\citeauthor[p.~397]{salgado_classical_2022})
	} % \citetitle

	Let
	\begin{math}
		f\in
		\continuous
	\end{math}.
	We seek calculate an approximation of
	\begin{equation*}
		I^{\openinterval}
		\left[
			f
			\right]\coloneqq
		\int_{a}^{b}
		f\left(x\right)
		\dl x.
	\end{equation*}

	Suppose that
	\begin{math}
		g\in
		\continuous
	\end{math},
	whose antiderivative is simply obtained, and
	\begin{math}
		{\left\|f-g\right\|}_{\infty}<
		\varepsilon
	\end{math}.
	Then,
	\begin{equation*}
		\left|
		\int_{a}^{b}
		f\left(x\right)
		\dl x-
		\int\limits_{a}^{b}
		g\left(x\right)
		\dl x
		\right|\leq
		\varepsilon
		\left(b-a\right).
	\end{equation*}

	\begin{definition}[Nodal set]
		Let
		\begin{math}
			\interval\subset
			\mathbb{R}
		\end{math}.
		$X$ is called a \emph{nodal set} of size $n+1\in\mathbb{N}$
		iff
		\begin{math}
			\nodalset\subset
			\interval
		\end{math}
		is a set of distinct elements.
		The elements of $X$, $x_{i}$ are called \emph{nodes}.
	\end{definition}

	\begin{definition}[Interpolating polynomial]
		Suppose that
		\begin{math}
			\nodalset\subset
			\interval
		\end{math}
		is a nodal set and
		\begin{math}
			f\colon
			\interval\to
			\mathbb{R}
		\end{math}
		is a function.
		The function
		\begin{math}
			I\colon
			\interval\to
			\mathbb{R}
		\end{math}
		is called an \emph{interpolant of} $f$ subordinate to $X$ iff
		\begin{math}
			\forall i=0,\dotsc,n:
			I\left(x_{i}\right)=
			f\left(x_{i}\right)
		\end{math},
		we write
		\begin{math}
			I\left(X\right)=
			f\left(X\right)
		\end{math}.
	\end{definition}
\end{frame}

\begin{frame}
	% TODO: Interpolant
	% \begin{definition}[Interpolant]
	%     Suppose
	%     \begin{math}
	%         Y=
	%         {\left\{y_{i}\right\}}_{i=0}^{n}\subset
	%         \mathbb{R}
	%     \end{math}
	%     is a set of not necessarily distinct points.
	%     Define the set of ordered pairs
	%     \begin{equation*}
	%         O=
	%         \left\{
	%         \left(x_{i},y_{i}\right)\mid
	%         x_{i}\in X,
	%         y_{i}\in Y,
	%         \forall i=0,\dotsc,n
	%         \right\}.
	%     \end{equation*}
	%     We say that $I$ is an \emph{interpolant} of $O$ iff
	%     \begin{math}
	%         \forall i=0,\dotsc,n:
	%         I\left(x_{i}\right)=
	%         y_{i}
	%     \end{math},
	%     we write
	%     \begin{math}
	%         I\left(X\right)=
	%         Y
	%     \end{math}.
	%     If the interpolant $I$ is a polynomial, it is called an
	%     \emph{interpolating polynomial}.
	% \end{definition}

	\begin{theorem}[existence and uniqueness]
		Suppose that
		\begin{math}
			\nodalset\subset
			\interval
		\end{math}
		is a nodal set
		and
		\begin{math}
			Y=
			{\left\{
			y_{i}
			\right\}}_{i=0}^{n}\subset
			\mathbb{R}
		\end{math}.
		There is a unique polynomial $p\in\mathbb{P}_{n}$ with the
		property that $p\left(X\right)=Y$.
	\end{theorem}

	% TODO: Interpolation operator
	% \begin{definition}[Interpolation operator]
	%     Suppose that
	%     \begin{math}
	%         \nodalset\subset
	%         \interval
	%     \end{math}.
	%     The \emph{interpolation operation} subordinate to $X$,
	%     denoted
	%     \begin{equation*}
	%         \mathcal{I}_{X}\colon
	%         \continuous\to
	%         \mathbb{P}_{n},
	%     \end{equation*}
	%     is defined as follows: for
	%     \begin{math}
	%         f\in
	%         \continuous
	%     \end{math},
	%     \begin{math}
	%         \mathcal{I}_{X}
	%         \left[f\right]\in
	%         \mathbb{P}_{n}
	%     \end{math}
	%     is the unique interpolating polynomial satisfying
	%     \begin{math}
	%         \mathcal{I}_{X}
	%         \left[f\right]
	%         \left(X\right)=
	%         f\left(X\right)
	%     \end{math}.
	% \end{definition}

	\begin{definition}[Lagrange nodal basis]
		Suppose that
		\begin{math}
			\nodalset\subset
			\interval
		\end{math}
		is a nodal set.
		The \emph{Lagrange nodal basis} subordinate to $X$ is the set
		of polynomials
		\begin{math}
			\mathcal{L}_{X}=
			{\left\{
			L_{\ell}
			\right\}}_{\ell=0}^{n}\subset
			\mathbb{P}_{n}
		\end{math}
		defined via
		\begin{equation*}
			L_{\ell}
			\left(x\right)=
			\prod\limits_{\substack{i=0\\i\neq\ell}}^{n}
			\dfrac{x-x_{i}}{x_{\ell}-x_{i}}.
		\end{equation*}
	\end{definition}

	% TODO: Properties of \mathcal{L}_{X}

	% TODO: interpolating polynomial

	\begin{definition}[Lagrange interpolating polynomial]
		Suppose that
		\begin{math}
			\nodalset\subset
			\interval
		\end{math}
		is a nodal set,
		\begin{math}
			\mathcal{L}_{X}=
			{\left\{L_{i}\right\}}_{i=0}^{n}
			\subset
			\mathbb{P}_{n}
		\end{math}
		is the Lagrange nodal basis subordinate to $X$, and
		\begin{math}
			f\colon
			\interval\to
			\mathbb{R}
		\end{math}.
		The \emph{Lagrange interpolating polynomial} of the function
		$f$, subordinate to the nodal set $X$, is the polynomial
		\begin{equation*}
			p\left(x\right)=
			\sum_{i=0}^{n}
			f\left(x_{i}\right)
			L_{i}\left(x\right)\in
			\mathbb{P}_{n}.
		\end{equation*}
	\end{definition}
\end{frame}

\begin{frame}
	Suppose that
	\begin{math}
		\nodalset\subset
		\interval
	\end{math}
	is a nodal set and $p\in\mathbb{P}_{n}$ is the unique Lagrange
	interpolating polynomial of $f$ subordinate to $X$.
	Then
	\begin{equation*}
		\forall i=0,\dotsc,n:
		f\left(x_{i}\right)=
		p\left(x_{i}\right)
	\end{equation*}
	and
	\begin{equation*}
		\forall x\in
		\interval:
		f\left(x\right)=
		p\left(x\right)+
		E\left(x\right),
	\end{equation*}
	where $E$ is an expression of the interpolation error.
	Then
	\begin{equation*}
		\int_{a}^{b}
		f\left(x\right)
		\dl x=
		\int_{a}^{b}
		p\left(x\right)
		\dl x+
		\int_{a}^{b}
		E\left(x\right)
		\dl x.
	\end{equation*}
	But
	\begin{equation*}
		\int_{a}^{b}
		p\left(x\right)
		\dl x=
		\int_{a}^{b}
		\sum_{i=0}^{n}
		f\left(x_{i}\right)
		L_{i}\left(x\right)=
		\sum_{i=0}^{n}
		f\left(x_{i}\right)
		\int_{a}^{b}
		L_{i}\left(x\right)=
		\sum_{i=0}^{n}
		f\left(x_{i}\right)
		\beta_{i},
	\end{equation*}
	where $L_{i}\in\mathbb{P}_{n}$ is the $i$th Lagrange nodal basis
	element and $\beta_{i}$ is its definite integral:
	\begin{equation*}
		\beta_{i}=
		\int_{a}^{b}
		L_{i}\left(x\right)
		\dl x.
	\end{equation*}
	The expression
	\begin{math}
		\sum_{i=0}^{n}
		f\left(x_{i}\right)
		\beta_{i}
	\end{math}
	is a typical numerical integration formula
	\begin{equation*}
		\left|
		\int_{a}^{b}
		f\left(x\right)
		\dl x-
		\sum_{i=0}^{n}
		f\left(x_{i}\right)
		\beta_{i}
		\right|=
		\left|
		\int_{a}^{b}
		E\left(x\right)
		\dl x
		\right|\leq
		\int_{a}^{b}
		\left|
		E\left(x\right)
		\right|
		\dl x.
	\end{equation*}
\end{frame}

\begin{frame}
	The \emph{quadrature weights}, $\beta_{i}$, depends only on the
	positions of the nodes within $\interval$, as well as the
	interval $\interval$ itself.
	% Suppose that $n=1$, $x_{0}=a$ and $x_{1}=b$.
	% Then
	% \begin{equation*}
	%     p\left(x\right)=
	%     f\left(a\right)
	%     \dfrac{x-b}{a-b}+
	%     f\left(b\right)
	%     \dfrac{x-a}{b-a}
	% \end{equation*}
	% and
	% \begin{equation*}
	%     \int_{a}^{b}
	%     p\left(x\right)
	%     \dl x=
	%     \dfrac{f\left(a\right)}{a-b}
	%     \int_{a}^{b}
	%     \left(x-b\right)
	%     \dl x+
	%     \dfrac{f\left(b\right)}{b-a}
	%     \int_{a}^{b}
	%     \left(x-a\right)
	%     \dl x=
	%     \dfrac{b-a}{2}
	%     \left(f\left(a\right)+f\left(b\right)\right).
	% \end{equation*}
	We will consider the approximation of a weight integral
	\begin{equation*}
		I^{\openinterval}_{w}
		\left[f\right]\coloneqq
		\int_{a}^{b}
		f\left(x\right)
		w\left(x\right)
		\dl x.
	\end{equation*}

	\begin{definition}[Quadrature rule]
		Suppose that $n,r\in\mathbb{N}_{0}$,
		$w$ is a weight function on
		\begin{math}
			\interval\subset
			\mathbb{R}
		\end{math},
		$h=b-a>0$, and
		\begin{math}
			f\in
			C^{r}\left(
			\interval
			\right)
		\end{math}.
		The expression
		\begin{equation*}
			Q^{\openinterval}_{w,r}
			\left[f\right]=
			\sum_{i=0}^{r}
			\sum_{j=0}^{n}
			\beta_{i,j}
			f^{\left(i\right)}
			\left(x_{j}\right)=
			\sum_{j=0}^{n}
			\left(
			\beta_{0,j}
			f\left(x_{j}\right)+
			\beta_{1,j}
			f^{\prime}
			\left(x_{j}\right)+
			\cdots+
			\beta_{r,j}
			f^{\left(r\right)}
			\left(x_{j}\right)
			\right),
		\end{equation*}
		where
		\begin{equation*}
			\forall i\in
			\left\{0,\dotsc,r\right\}:
			\forall j\in
			\left\{0,\dotsc,n\right\}:
			\beta_{i,j}=
			h^{i+1}
			\widehat{\beta}_{i,j}
		\end{equation*}
		and
		\begin{equation*}
			\forall j\in
			\left\{0,\dotsc,n\right\}:
			x_{j}=
			a+
			h\cdot
			\widehat{x}_{j},
		\end{equation*}
		is called a \emph{quadrature rule of degree $r$} with
		\emph{intrinsic nodes}
		\begin{math}
			\widehat{X}=
			\left\{
			\widehat{x}_{j}
			\right\}\subset
			\left[0,1\right]
		\end{math}
		and \emph{intrinsic weights}
		\begin{math}
			\left\{
			\beta_{i,j}
			\right\}
			\subset
			\mathbb{R}
		\end{math}
		are called the \emph{effective nodes} and
		\emph{effective weights}, respectively.
	\end{definition}
	A quadrature rule of degree $r=0$ is called a
	\emph{simple quadrature rule}, and we simplify the notation
	by writing $\beta_{j}=\beta_{0,j}$ and
	\begin{equation*}
		Q^{\openinterval}_{w,r}
		\left[f\right]=
		\sum_{j=0}^{n}
		\beta_{j}
		f^{\left(i\right)}
		\left(x_{j}\right).
	\end{equation*}
\end{frame}

\begin{frame}

	The \emph{quadrature rule error} is defined as
	\begin{equation*}
		E_{Q}
		\left[f\right]=
		I^{\openinterval}_{w}
		\left[f\right]-
		Q^{\openinterval}_{w,r}
		\left[f\right].
	\end{equation*}

	\begin{definition}[consistency]
		The quadrature rule is \emph{consistent of order at least}
		$m\in\mathbb{N}_{0}$ iff $E_{Q}\left[q\right]=0$ for all
		$q\in\mathbb{P}_{m}$.
		The quadrature rule is \emph{consistent of order exactly} $m$
		iff $E_{Q}\left[q\right]=0$ for all $q\in\mathbb{P}_{m}$;
		however, for some $r\in\mathbb{P}_{m+1}$,
		$E_{Q}\left[r\right]\neq 0$.
	\end{definition}

	\begin{definition}[interpolatory quadrature rule]
		Assume that $n\in\mathbb{N}_{0}$, $w$ is a weight function on
		$\interval\subset\mathbb{R}$, and $f\in\continuous$.
		Suppose that $\nodalset\subset\interval$ is a nodal set and
		$p\in\mathbb{P}_{n}$ is the unique Lagrange Interpolating
		polynomial of $f$ subordinate to $X$, with
		\begin{equation*}
			p\left(x\right)=
			\sum_{j=0}^{n}
			f\left(x_{j}\right)
			L_{j}\left(x\right),
		\end{equation*}
		where $L_{j}\in\mathbb{P}_{n}$ is the $j$th Lagrange nodal
		basis element.
	\end{definition}
	The expression
	\begin{equation*}
		Q^{\openinterval}_{w}
		\left[f\right]=
		\sum_{j=0}^{n}
		f\left(x_{j}\right)
		\beta_{j},
	\end{equation*}
	where
	\begin{equation*}
		\beta_{j}=
		\int_{a}^{b}
		L_{j}\left(x\right)
		w\left(x\right)
		\dl x,
	\end{equation*}
	is called an \emph{interpolatory quadrature rule subordinate to $X$ of Lagrange type}
	for approximating
	\begin{math}
		I^{\openinterval}_{w}
		\left[f\right]
	\end{math}.
	% TODO: Hermite
	% Suppose that $f\in $
\end{frame}

\begin{frame}
	\begin{theorem}[existence and uniqueness]
		Suppose that $\nodalset\subset\mathbb{R}$ is a nodal set.
		There exists uniqueness weights $\left\{\beta_{j}\right\}_{j=0}^{n}$
		such that
		\begin{equation*}
			\forall q\in\mathbb{P}_{n}:
			\int_{a}^{b}
			q\left(x\right)
			w\left(x\right)
			\dl x=
			\sum_{j=0}^{n}
			\beta_{j}
			q\left(x_{j}\right),
		\end{equation*}
		or, equivalently,
		\begin{equation*}
			\forall q\in\mathbb{P}_{n}:
			E_{Q}\left[q\right]=0.
		\end{equation*}
		Moreover, these weights are given by
		\begin{equation*}
			\forall j\in\left\{0,\dotsc,n\right\}:
			\beta_{j}=
			\int_{a}^{b}
			L_{j}\left(x\right)
			w\left(x\right)
			\dl x,
		\end{equation*}
		where $L_{j}$ is the $j$th Lagrange nodal basis polynomial subject to $X$.
	\end{theorem}

	% TODO: Remark
	\begin{theorem}[consistency]
		The last result shows that the simple quadrature rule,
		\begin{equation*}
			Q^{\openinterval}_{w}
			\left[f\right]=
			\sum_{j=0}^{n}
			\beta_{i}
			f\left(x_{i}\right),
		\end{equation*}
		is consistent of order at least $n$ iff it is a quadrature
		rule of Lagrange type.
	\end{theorem}
\end{frame}

\begin{frame}
	\begin{theorem}[Error estimate]
		Suppose that $n\in\mathbb{N}_{0}$, $w$ is a weight function
		on
		\begin{math}
			\interval\subset
			\mathbb{R}
		\end{math}.
		\begin{math}
			f\in C^{n+1}
			\left(
			\interval
			\right)
		\end{math},
		and
		\begin{math}
			X=
			\left\{x_{i}\right\}_{i=0}^{n}\subset
			\interval
		\end{math}
		is a nodal set.
		Suppose that
		\begin{math}
			Q^{\openinterval}_{w}
			\left[f\right]
		\end{math}
		is the interpolatory quadrature rule subordinate to $X$ of Lagrange type.
		Then,
		\begin{equation*}
			\left|
			E_{Q}
			\left[f\right]
			\right|\leq
			\dfrac{M_{n+1}}{\left(n+1\right)!}
			\int_{a}^{b}
			\left|
			\omega_{n+1}
			\left(x\right)
			\right|
			w\left(x\right)
			\dl x,
		\end{equation*}
		where
		\begin{equation*}
			\omega_{n+1}
			\left(x\right)=
			\prod\limits_{j=0}^{n}
			\left(x-x_{j}\right)
		\end{equation*}
		and
		\begin{equation*}
			M_{n+1}=
			{\left\|
			f^{\left(n+1\right)}
			\right\|}_{\infty}.
		\end{equation*}
		Consequently, an interpolatory quadrature rule subordinate to
		$X$ of Lagrange type is consistent of order at least $n$.
	\end{theorem}
	% Error hermite
	% \begin{theorem}[error estimate]
	%     Suppose that $n\in\mathbb{N}_{0}$, $w$ is a weight function on
	%     $\interval\subset\mathbb{R}$, $f\in C^{\left(2n+2\right)}\left(\interval\right)$,
	%     and $\nodalset\subset\interval$ is a nodal set.
	%     Suppose that
	%     \begin{math}
	%         Q^{\openinterval}_{w,1}
	%         \left[f\right]
	%     \end{math}
	%     is the interpolatory quadrature rule subordinate to $X$
	% \end{theorem}
	\begin{definition}[characteristic function]
		Suppose that $B\subset\mathbb{R}$.
		The \emph{characteristic function} of $B$ is the function
		\begin{equation*}
			\chi_{B}\left(t\right)=
			\begin{cases}
				1, & t\in B,                    \\
				0, & t\in\mathbb{R}\setminus B.
			\end{cases}
		\end{equation*}
	\end{definition}
\end{frame}

\begin{frame}
	% TODO: lemma
	\begin{theorem}[kernel]
		Suppose that $r\in\mathbb{N}_{0}$ and $m\in\mathbb{N}$, with
		$m>r$.
		Define the function
		\begin{math}
			k_{m}\colon
			\interval\times\interval\to
			\mathbb{R}
		\end{math}
		via
		\begin{equation*}
			k_{m}\left(x,y\right)=
				{\left(x-y\right)}^{m}
			\xi_{\left[a,x\right]}\left(y\right)=
			\begin{cases}
				\left(x-y\right)^{m}, & a\leq y\leq x\leq b, \\
				0,                    & a\leq x< y\leq b.
			\end{cases}
		\end{equation*}
		Then, for each $i\in\left\{0,\dotsc,r\right\}$,
		\begin{equation*}
			\diffp[i]{k_{m}}{x}\in
			C\left(\interval\times\interval\right)
		\end{equation*}
		and
		\begin{equation*}
			\diffp[i]{k_{m}\left(x,y\right)}{x}=
			\begin{cases}
				\prod\limits_{k=0}^{i-1}
				\left(m-k\right)
				{\left(x-y\right)}^{m-i}, & a\leq y\leq x\leq b, \\
				0,                        & a\leq x< y\leq b.
			\end{cases}
		\end{equation*}
	\end{theorem}
\end{frame}

\begin{frame}
	\begin{theorem}[Peano Kernel Theorem]
		Suppose that $r\in\mathbb{N}_{0}$, $m\in\mathbb{N}$, with $m>n$,
		$w$ is a weight function on $\left[a,b\right]\subset\mathbb{R}$, and
		$f\in C^{m+1}\left(\interval\right)$.
		Assume that
		\begin{math}
			Q^{\openinterval}_{w,r}
			\left[f\right]
		\end{math}
		is a quadrature rule of degree $r$,
		that is consistent of order at least $m$.
		Let the function
		\begin{math}
			k_{m}\colon\interval\times\interval\to\mathbb{R}
		\end{math}.
		Set
		\begin{equation*}
			K_{m}\left(y\right)=
			E_{Q}\left[k_{m}\left(\cdot, y\right)\right]=
			\int_{a}^{b}
			k_{m}\left(x,y\right)w\left(x\right)\dl x-
			\sum_{j=0}^{n}
			\sum_{i=0}^{r}
			\beta_{i,j}
			\diffp[i]{k_{m}\left(x_{j},y\right)}{x}.
		\end{equation*}
		Then the quadrature error satisfies
		\begin{equation*}
			E_{Q}\left[f\right]=
			\dfrac{1}{m!}
			\int_{a}^{b}
			f^{\left(m+1\right)}\left(y\right)
			K_{m}\left(y\right)
			\dl y.
		\end{equation*}
		The function $K_{m}\left(y\right)$ is called the \emph{Peano Kernel}.
	\end{theorem}

	% TODO: remark
	% \begin{theorem}[Peano Kernel]
	%     Re.
	% \end{theorem}
	\begin{theorem}[quadrature error stability]
		We have
		\begin{equation*}
			\left|
			E_{Q}
			\left[f\right]
			\right|\leq
			\dfrac{1}{m!}
			{\left\|
				f^{\left(m+1\right)}
				\right\|}_{\infty}
			{\left\|
				K_{m}
				\right\|}_{1}.
		\end{equation*}
		Since ${\left\|K_{m}\right\|}_{1}<\infty$, there is a
		constant $C>0$ that may depend on the size of the interval
		but is independent of $f$ such that
		\begin{equation*}
			\left|
			E_{Q}
			\left[f\right]
			\right|\leq
			C{\left\|
					f^{\left(m+1\right)}
					\right\|}_{\infty}.
		\end{equation*}
	\end{theorem}
\end{frame}

\begin{frame}
	% TODO: Theorem
	\begin{theorem}[constant sign]
		If $K_{m}$ does not change sign in $\interval$, then
		\begin{equation*}
			E_{Q}\left[f\right]=
			\dfrac{f^{\left(m+1\right)\left(\xi\right)}}{m!}
			\int_{a}^{b}K_{m}
			\dl y
		\end{equation*}
		for some $\xi\in\interval$.
		Furthermore, we have the simple representation for the error
		\begin{equation*}
			E_{Q}\left[f\right]=
			\dfrac{E_{Q}\left[x^{m+1}\right]}{\left(m+1\right)!}
			f^{\left(m+1\right)}\left(\xi\right)
		\end{equation*}
		for some $\xi\in\interval$, where $E_{Q}\left[x^{m+1}\right]$
		is the quadrature error for the function $x\mapsto x^{m+1}$.
	\end{theorem}

	\begin{definition}[Peano Kernel]
		Let $M_{1}=\chi_{\interval}$.
		For $k\in\mathbb{N}$ with $k\geq 2$, set
		\begin{equation*}
			M_{k}\left(x\right)=
			\int_{\mathbb{R}}
			M_{k-1}\left(x-y\right)M_{1}\left(y\right)
			\dl y.
		\end{equation*}
		For $k\in\mathbb{N}$ and $h>0$, we define
		\begin{equation*}
			M_{k}\left(x,h\right)=
			\dfrac{1}{h}
			M_{k}
			\left(\dfrac{x}{h}\right).
		\end{equation*}
	\end{definition}
	% TODO: lemma
	% TODO: integral representation
	% TODO: modulus of smoothness
	% TODO: scaling argument
\end{frame}

\begin{frame}
	\begin{definition}[Closed Newton-Cotes quadrature rule]
		Suppose that $w$ is a weight function on
		\begin{math}
			\interval\subset
			\mathbb{R}
		\end{math}
		and $n\in\mathbb{N}$.
		Set $h=b-a>0$ and $\hslash=\frac{h}{n}$.
		Suppose that, for the simple quadrature rule, the nodal set
		$\nodalset\subset\interval$ is defined by
		\begin{equation*}
			x_{j}=
			a+j\hslash,\quad
			j\in\left\{0,\dotsc,n\right\}.
		\end{equation*}
		The resulting method, denoting $Q_{n}\left[f\right]$, is
		called a \emph{closed Newton-Cotes quadrature rule of order $n$}.
	\end{definition}

	\begin{example}[Newton-Cotes quadrature rules of order $n=1,2$ and weight function $w\equiv 1$ on $\interval$]
		\begin{equation*}
			x_{j}=
			a+h\widehat{x}_{j},\quad
			\beta_{j}=h\widehat{\beta}_{j},\quad
			h=b-a.
		\end{equation*}

		\begin{table}[ht!]
			\centering
			\begin{tabular}{ccccc}
				\hline
				$n$ & rule                                                         & $\widehat{x}_{j}$ & $\widehat{\beta}_{j}$
				    & Error Formula                                                                                                              \\
				$1$ & Trapezoidal                                                  & $0,1$             & $\frac{1}{2}, \frac{1}{2}$
				    & $-\frac{1}{12}\hslash^{3}f^{\left(2\right)}\left(\xi\right)$                                                               \\
				$2$ & Simpson's                                                    & $0,\frac{1}{2},1$ & $\frac{1}{6}, \frac{4}{6}, \frac{1}{6}$
				    & $-\frac{1}{90}\hslash^{5}f^{\left(4\right)}\left(\xi\right)$                                                               \\
				\hline
			\end{tabular}
		\end{table}
		For $n=2$, we observe the phenomenon of
		\alert{super-convergence}, i.e., a higher than expected convergence.

		For example, consider the case $n=3$, Simpson's $\frac{3}{8}$
		rule, on the reference interval $\left[0,1\right]$.
		The second Lagrange nodal basis element is
		\begin{equation*}
			\widehat{L}_{1}\left(x\right)=
			\dfrac{
				x
				\left(x-\frac{2}{3}\right)
				\left(x-1\right)
			}{
				\frac{1}{3}
				\left(\frac{1}{3}-\frac{2}{3}\right)
				\left(\frac{1}{3}-1\right)
			}=
			\dfrac{27}{2}
			\left(
			x^{3}-\dfrac{5}{3}x^{2}+\dfrac{2}{3}x
			\right).
		\end{equation*}
		Then,
		\begin{equation*}
			\widehat{\beta}_{1}=
			\int_{0}^{1}
			\widehat{L}_{1}\left(x\right)
			\dl x=
			\dfrac{27}{2}
			{\left.
				\left(
				\dfrac{1}{4}x^{4}-
				\dfrac{5}{9}x^{3}+
				\dfrac{1}{3}x^{2}
				\right)
				\right|}_{x=0}^{x=1}=
			\dfrac{3}{8}.
		\end{equation*}
	\end{example}
\end{frame}

\begin{frame}
	\begin{theorem}[Error estimate]
		Let
		\begin{math}
			\interval\subset
			\mathbb{R}
		\end{math}.
		Suppose that $Q^{\openinterval}_{n}\left[f\right]$ is a
		closed Newton-Cotes quadrature rule of order
		$n\in\mathbb{N}$.
		Then, the order of quadrature rule is consistent of order at
		least $n$.
		Moreover, if $f\in C^{n+1}\left(\interval\right)$, then
		\begin{equation*}
			\left|
			E_{Q_{n}}
			\left[f\right]
			\right|\leq
			C
			h^{n+2}
			{\left\|f^{\left(n+1\right)}\right\|}_{\infty},
		\end{equation*}
		where $h=b-a$ and $C>0$ is independent of $h$ and $f$.
	\end{theorem}

	\begin{theorem}[Integral Mean Value Theorem]
		Suppose that
		\begin{math}
			-\infty<a<b<\infty
		\end{math},
		\begin{math}
			f\in
			\continuous
		\end{math},
		and
		\begin{math}
			g\in
			\mathcal{R}
			\openinterval
		\end{math}.
		Furthermore, suppose that
		\begin{math}
			\forall x\in\interval:
			g\left(x\right)\geq0
		\end{math}.
		Then there exists a point $\xi\in\interval$ such that
		\begin{equation*}
			\int_{a}^{b}
			f\left(x\right)
			g\left(x\right)
			\dl x=
			f\left(\xi\right)
			\int_{a}^{b}
			g\left(x\right)
			\dl x.
		\end{equation*}

		Thus, if
		\begin{math}
			\forall x\in
			\interval:
			g\left(x\right)=
			1
		\end{math},
		there exists a point $\xi\in\interval$ such that
		\begin{equation*}
			f\left(\xi\right)=
			\dfrac{1}{b-a}
			\int_{a}^{b}
			f\left(x\right)
			\dl x.
		\end{equation*}
	\end{theorem}
\end{frame}
