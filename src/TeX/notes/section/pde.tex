\section{Partial Differential Equations}

\begin{frame}
	\frametitle{
		\secname~(\citeauthor[p.~2]{choksi_partial_2022})
	} % \citetitle

	\begin{definition}[Ecuación Diferencial Parcial (EDP)]
		Es una ecuación que involucra una \emph{función desconocida} $u$
		y sus derivadas parciales junto con las variables independientes.
		Se escribe como
		\begin{equation}
			\mathcal{L}
			\left(
			\text{variables independientes},
			u,
			\text{derivadas de $u$}
			\right)
			=0.
			\label{eq:pde}
		\end{equation}
	\end{definition}

	\begin{definition}[Dominio]
		Un dominio $\Omega$ es un subconjunto de $\mathbb{R}^{d}$ abierto
		y conexo que tiene frontera lineal a trozos de clase $C^{1}$.
	\end{definition}

	En lo sucesivo $\Omega$ siempre será un dominio.

	\begin{definition}[Solución clásica de la EDP]
		Es una función $u\colon\Omega\to\mathbb{R}$ suficientemente suave
		que satisface~\eqref{eq:pde} para cualquier $x\in\Omega$.
	\end{definition}

	\begin{definition}[Condiciones auxiliares]
		Una \emph{condición auxiliar} en una solución general es una
		igualdad que especifica el valor de la función desconocida en un
		subconjunto de $\Omega$.
	\end{definition}
\end{frame}

\begin{frame}
	\begin{definition}[Problema de Valor Inicial (PVI)]
		Sea
		\begin{math}
			u\colon
			\Omega\times\left[0,T\right]
			\to\mathbb{R}
		\end{math}
		una solución de~\eqref{eq:pde}.
		Un \emph{problema de valor inicial} es una ecuación diferencial
		junto con un conjunto de condiciones auxiliares que especifican
		la solución y/o sus derivadas en $t=0$.
	\end{definition}

	\begin{example}[PVI de la ecuación de difusión]
		\begin{equation}
			\begin{cases}
				\difcp{u}{t}-
				\alpha^{2}
				\difc.L.{u}{}=0 &
				\text{ para }
				\left(x,t\right)\in
				\Omega\times\left[0,T\right]. \\
				u
				\left(x,0\right)=
				\phi
				\left(x\right),\quad
				\difcp{u}{t}
				\left(x,0\right)=
				\psi
				\left(x\right)  &
				\text{ para }
				x\in\Omega.
			\end{cases}\label{eq:diffusion}
		\end{equation}
	\end{example}

	\begin{definition}[Solución clásica]
		\begin{math}
			u\colon
			\Omega
			\times\left[0,T\right]\to
			\mathbb{R}
		\end{math}
		es la
		\emph{solución clásica} de~\eqref{eq:diffusion} sii
		\begin{math}
			\forall t>0:
			u\left(\cdot,t\right)\in C^{2}\left(\Omega\right),
			\forall x\in\Omega:
			u\left(x,\cdot\right)\in C\left(\left[0,T\right]\right),
		\end{math}
		la ecuación y la condición inicial se cumplen puntualmente, y
		cuando $t\downarrow0$ tenemos que
		\begin{math}
			\forall x\in\Omega:
			u\left(x,t\right)\to u_{0}\left(x\right)
		\end{math}.
	\end{definition}

	\begin{theorem}[Existencia]
		Sea $u_{0}\in C_{b}\left(\Omega\right)\cap L^{2}\left(\Omega\right)$.
		Entonces, $u$ es una solución clásica de~\eqref{eq:diffusion}.
		\begin{equation*}
			u\left(x,t\right)\to u_{0}\left(x_{0}\right),\qquad
			\left(x,t\right)\to\left(x_{0},0\right)
		\end{equation*}
	\end{theorem}
\end{frame}
