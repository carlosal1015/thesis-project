\begin{titlepage}
	\begin{center}
		\vspace*{-18mm}
		{\LARGE \textbf{Universidad Nacional de Ingeniería}}\\ % 18pt
		\vspace{5mm}
		{\Large \textbf{\@facultad}}\\ % 14pt
		\vspace{6.5mm}
		\begin{figure}[h]
			\centering
			\includegraphics[width=.28\paperwidth]{E_IMAGENES/0_Caratula/UNI_LOGO1.pdf}
		\end{figure}
		\vspace{1mm}
		{\Large \textbf{PROYECTO DE TESIS I} }\\%11pt
		\vspace{5mm}

		\onehalfspacing  % Espaciamiento 1.5
		{\Large \textbf{``{\@titlecaratula}''} }\\ % 11pt

		\singlespacing  % Fin del espaciamiento 1.5

		\vspace{5mm}
		% {\large \textbf{PARA OBTENER EL TÍTULO PROFESIONAL DE {\@grado} } }\\ % 12pt
		% \vspace{10mm}
		{\large \textbf{Elaborado por:} }\\ % 12pt
		\vspace{5mm}
		{\large \textbf{\@authorcaratula} }\\ % 12pt
		\vspace{10mm}
		{\large \textbf{Asesor:} }\\
		\vspace{5mm}
		{\large \textbf{\@asesor} }\\
		\vspace{10mm}
		{\large \textbf{Asesor Externo:} }\\
		\vspace{5mm}
		{\large \textbf{\@asesorexterno} }\\
		\vspace{10mm}
		{\large \textbf{LIMA - PERÚ} }\\
		\vspace{5mm}
		{\large \textbf{\@yyearr} }
	\end{center}
	\vfill
\end{titlepage}
% TODO: Redacción en tercera persona
% TODO: Eg.En esta parte del trabajo presentamos que sirve de base para desarrollar
% TODO: Esta sustenado por este teorema que trata sobre
% TODO: Antes de colocar la solucion ratio de LW, agregar una introducción
% TODO: entropía
