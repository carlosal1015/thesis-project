\usepackage[natbibapa]{apacite} %Agregar formato de citación APA
\bibliographystyle{apacite}
\setlength{\bibsep}{5pt plus 0.3ex} %Espaciamiento en la bibliografía
\setcounter{secnumdepth}{3} %Numera los subsubsecciones
% \usepackage[T1]{fontenc}
\usepackage[spanish,es-tabla]{babel} %reemplaza "cuadro" por "tabla"
\decimalpoint % Cambia coma por punto
\usepackage{mathptmx}
% \usepackage[cmintegrals]{newtxmath}
\usepackage{layouts} % Saber ancho de hoja.
\usepackage{fontspec}
\usepackage[ISO]{diffcoeff}
\usepackage{amsthm}
\usepackage{mathtools}
% \usepackage{bm}
\usepackage{amssymb}
\let\mathbbalt\mathbb
\usepackage{graphicx}
\usepackage{changepage} % Agregar identación
\setmainfont{Arial}
\usepackage{setspace}
\usepackage[left=4 cm,right=3 cm,top=3 cm,bottom=3.5 cm]{geometry}
\usepackage{fancyhdr} % Encabezado y Pie de página (1)
\pagestyle{fancy} % Encabezado y Pie de página (2)
\usepackage{lipsum} % Crear texto RAMDOM
\renewcommand{\labelitemi}{$\bullet$} % Circulos para viñetas
\usepackage{titlesec} % Titulos de SECCIONES
\usepackage{tocloft} % Titulos de ÍNDICES
\usepackage[colorlinks=true,linkcolor=negro,citecolor = negro]{hyperref}
\usepackage[mathbf=sym]{unicode-math} % Mantener fuentes matemáticas
\removenolimits{\int} % https://tex.stackexchange.com/a/103925
\let\mathbb\mathbbalt
\makeatletter  % Comando para REDUCIR ERRORES
\usepackage{multirow} % Agregar TABLAS 
\usepackage{array} % Dar formato a las TABLAS
\usepackage{subcaption} % Insertar SubImagenes
\usepackage{tikz} % Diagrama de Flujo
\usetikzlibrary{calc,positioning,shapes.geometric,shapes.symbols,shapes.misc}

% Insertar formas para diagramas de flujo
\tikzstyle{startstop} = [rectangle, rounded corners, minimum width=3cm, minimum height=0.5cm,text centered, draw=black]
\tikzstyle{io} = [trapezium, trapezium left angle=70, trapezium right angle=110, minimum width=3cm, minimum height=0.5cm, text centered, text width=3cm, draw=black]
\tikzstyle{process} = [rectangle, minimum width=3cm, minimum height=0.5cm, text centered, text width=4cm, draw=black]
\tikzstyle{decision} = [diamond, minimum width=3cm, minimum height=0.5cm, text centered, draw=black]
\tikzstyle{loop} = [chamfered rectangle,chamfered rectangle xsep=2cm,draw=black]
\tikzstyle{arrow} = [->,>=stealth]
\tikzstyle{line}=[draw]

\usepackage{listings}
\usepackage{color}

\AtBeginDocument{\let\setminus\smallsetminus}
\DeclareMathAlphabet{\mathcal}{OMS}{cmsy}{m}{n}

%New colors defined below
\definecolor{codegreen}{rgb}{0,0.6,0}
\definecolor{codegray}{rgb}{0.5,0.5,0.5}
\definecolor{codepurple}{rgb}{0.2,0,1}
\definecolor{codeRojo}{rgb}{0.7,0,0.3}
\definecolor{backcolour}{rgb}{1.0, 1.0, 1.0}

%Code listing style named "mystyle"
\lstdefinestyle{mystyle}{
  backgroundcolor=\color{backcolour},
  commentstyle=\color{codegreen},
  keywordstyle=\color{codeRojo},
  numberstyle=\tiny\color{codegray},
  stringstyle=\color{codepurple},
  basicstyle=\footnotesize,
  breakatwhitespace=false,
  breaklines=true,
  captionpos=b,
  keepspaces=true,
  numbers=left,
  numbersep=5pt,
  showspaces=false,
  showstringspaces=false,
  showtabs=false,
  tabsize=2
}

%"mystyle" code listing set
\lstset{style=mystyle}

\newenvironment{MyFont}{\fontfamily{ugm}\selectfont}{\par}

\usepackage{changepage} % Agregar espacio a Listing

% Centrado del título del ÍNDICE / LISTA DE FIGURAS / LISTA DE CUADROS

\renewcommand{\cfttoctitlefont}{\hfill \normalfont\normalsize\bfseries}
\renewcommand{\cftaftertoctitle}{\hfill}

\renewcommand{\cftlottitlefont}{\hfill\normalfont\normalsize\bfseries}
\renewcommand{\cftafterlottitle}{\hfill}

\renewcommand{\cftloftitlefont}{\hfill\normalfont\normalsize\bfseries}
\renewcommand{\cftafterloftitle}{\hfill}

% Formato de los CAPÍTULOS, SECCIONES Y SUBSECCIONES
\titleformat{\chapter}[block]{\normalfont\normalsize\bfseries}{CAPÍTULO \thechapter:}{0.5em}{\normalsize}
\titlespacing*{\chapter}{0pt}{-10 pt}{5pt}

\titleformat{\section}[block]{\normalfont\normalsize}{\thesection}{0.5em}{\normalsize}
\titlespacing*{\section}{0pt}{10 pt}{5 pt}

\titleformat{\subsection}[block]{\normalfont\normalsize}{\thesubsection}{0.5em}{\normalsize}
\titlespacing*{\subsection}{0pt}{10 pt}{5 pt}

\titleformat{\subsubsection}[block]{\normalfont\normalsize}{\thesubsubsection}{0.5em}{\normalsize}
\titlespacing*{\subsubsection}{0pt}{10 pt}{5 pt}

% Espaciado entre PÁRRAFOS y SANGRÍA
\setlength{\parskip}{5 pt}
\setlength{\parindent}{0cm}

% Dar formato a las TABLAS
\newcolumntype{M}[1]{>{\centering\arraybackslash}m{#1}}
\newcolumntype{L}[1]{>{\raggedright\arraybackslash}m{#1}}
\newcolumntype{R}[1]{>{\raggedleft\arraybackslash}m{#1}}
\newcolumntype{N}{@{}m{0pt}@{}}
\renewcommand{\arraystretch}{1.25}

% Cambiar titulo de bibliografía
\addto\captionsspanish{\renewcommand{\bibname}{\centering REFERENCIAS BIBLIOGRÁFICAS}}

% Posicionamiento vertical de TOC, LOT and LOF

\setlength{\cftbeforelottitleskip}{1pt}
\renewcommand{\cftafterlottitleskip}{12pt}

\setlength{\cftbeforeloftitleskip}{1pt}
\renewcommand{\cftafterloftitleskip}{12pt}

\setlength{\cftbeforetoctitleskip}{16pt}
\renewcommand{\cftaftertoctitleskip}{12 pt}

% Cambiando las etiquetas de las FIGURAS y TABLAS (Caption y autoref)
\addto\captionsspanish{\renewcommand{\figurename}{\footnotesize FIGURA N°}}
\addto\extrasspanish{\def\figureautorefname{Figura N°}}

\addto\captionsspanish{\renewcommand{\tablename}{\footnotesize TABLA N°}}
\addto\extrasspanish{\def\tableautorefname{Tabla N°}}

% Espaciamiento dentro del índice

\setlength{\cftbeforechapskip}{2mm}
\renewcommand\cftchapafterpnum{\vskip6pt}
\renewcommand\cftsecafterpnum{\vskip5pt}
\renewcommand\cftsubsecafterpnum{\vskip5pt}

%Agregar la palabra CAPITULO al TOC, FIGURA a LOF y TABLA al LOT

\renewcommand{\cftchappresnum}{CAPÍTULO}
\renewcommand{\cftchapaftersnum}{:}
\renewcommand{\cftchapnumwidth}{7em}

\renewcommand{\cftfigpresnum}{Figura N°}
%\renewcommand{\cftfigaftersnum}{:}
\renewcommand{\cftfignumwidth}{6.85 em}

\renewcommand{\cfttabpresnum}{Tabla N°}
%\renewcommand{\cftfigaftersnum}{:}
\renewcommand{\cfttabnumwidth}{6.5 em}

%Definición de COLORES
\definecolor{granate}{RGB}{113,22,16}
\definecolor{gris}{RGB}{154,153,157}
\definecolor{arena}{RGB}{230,217,170}
\definecolor{azul}{rgb}{0.03,0.15,0.4}
\definecolor{negro}{rgb}{0,0,0}

%Cambiando a Números Romanos los Capítulos
\renewcommand{\thechapter}{\Roman{chapter}}
\renewcommand{\theequation}{\arabic{chapter}.\arabic{equation}}
\renewcommand{\thesection}{\arabic{chapter}.\arabic{section}}
\renewcommand{\thetable}{\arabic{chapter}.\arabic{table}}
\renewcommand{\thefigure}{\arabic{chapter}.\arabic{figure}}

\providecommand{\averageconcentration}{\overline{u}\left(x,t\right)}
\providecommand{\inner}[2]{\left\langle #1, #2 \right\rangle}
\difdef{c}{L}{op-symbol=\mathop{}\!\mathbin\bigtriangleup}
\difdef{c}{A}{op-symbol=\mathop{}\!\mathbin\Box}


\theoremstyle{definition}
\newtheorem{theorem}{Teorema}
\newtheorem{definition}{Definición}
\newtheorem{example}{Ejemplo}
\newtheorem{proposition}{Proposición}

% Escribir los INFORMACIÓN PERSONAL Y DEL TRABAJO

%Autor para PIE DE PÁGINA (Respetar mayusculas y minisculas)
\author{Carlos Alonso Aznarán Laos}

%Autor para CARÁTULA (Siempre en mayuscula y sin saltos de linea)
\authorcaratula{CARLOS ALONSO AZNARÁN LAOS}

%Título en para el PIE DE PÁGINA (Agregar salto de línea de ser necesario)
% La selección de un esquema Upwind y Lax-Wendroff para la solución numérica de la ecuación advección-difusión
% Minimización del error en los métodos Upwind y Lax-Wendroff para la solución numérica de la ecuación advección-difusión % Evolutivos
% Estimación del Error en los Esquemas Upwind y Lax-Wendroff de la ecuación de Advección-Difusión lineal
\title{Estimación del Error en los Esquemas de Alta Resolulución de la ecuación de Advección-Difusión lineal}
%Título para CARÁTULA (Siempre en mayuscula y sin saltos de linea)
\titlecaratula{ESTIMACIÓN DEL ERROR EN LOS ESQUEMAS DE ALTA RESOLUCIÓN DE LA ECUACIÓN DE ADVECCIÓN-DIFUSIÓN LINEAL} % EVOLUTIVOS
%Nombre de la FACULTAD (Siempre en mayuscula)
\facultad{FACULTAD DE CIENCIAS}

%Para obtener el título profesional de ... (Siempre en mayúscula)
\grado{LICENCIADO EN MATEMÁTICA}

%Asesor para CARÁTULA (Siempre en mayuscula y sin saltos de linea)
\asesor{DR. JONATHAN ALFREDO MUNGUÍA LA COTERA}

% Año para la CARÁTULA
\yyearr{2024}