\cleardoublepage\phantomsection\addcontentsline{toc}{chapter}{\bf RESUMEN}
\chapter*{\centerline {RESUMEN}}
\markboth{RESUMEN}{}

En este proyecto de tesis I, presentamos la formulación matemática de
la ecuación diferencial parcial \emph{advección-difusión}.
Este modelo base permite estudiar la cinética de sistemas de
reacciones químicas y el transporte de contaminantes en fenómenos
meteorológicos.
Nuestro estudio inicia con la presentación de las formas
diferenciales e integrales de la ley de conservación y la deducción
de la ecuación de advección-difusión lineal unidimensional.
% y bidimensional
Introducimos los esquemas de discretización de primer y segundo orden
para espacio y tiempo.
Se estudia la consistencia, estabilidad y convergencia de los métodos
de primer orden Upwind (FOU), adelante en tiempo - centrado en
espacio (FTCS), Lax-Friedrichs, Leap-Frog, Lax-Wendroff.
.
Finalmente, se analiza los resultados de convergencia en las pruebas
numéricas sometidos a los esquemas presentados.