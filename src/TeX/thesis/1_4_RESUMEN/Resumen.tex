\cleardoublepage\phantomsection\addcontentsline{toc}{chapter}{\bf RESUMEN}
\chapter*{\centerline {RESUMEN}}
\markboth{RESUMEN}{}
% TODO: M'aximo 250 palabras.
Este proyecto de tesis I se ocupa de los esquemas numéricos de
alta resolución\footnote{
  Acuñado por~\citefullauthor{Harten1983} en el año~\citeyear{Harten1983}.
}
que emplea limitadores de pendiente para resolver el problema de
Cauchy de la ecuación diferencial parcial \emph{advección-difusión}:

\begin{equation}
  \difcp{u}{t}+
  \operatorname{div}
  \left(vu-D\difc.L.{u}{}\right)=
  0\text{ para }
  \left(x,t\right)\in
  \Omega\times\left[0,T\right].
  \label{eq:advection-diffusion}
\end{equation}

Si todos los términos están presentes, la ecuación~\eqref{eq:advection-diffusion}
es parábolica

\begin{table}[ht!]
  \centering
  \begin{tabular}{lcc}
    Parabólica  &
    \begin{math}
      \difcp{u}{t}+
      \operatorname{div}
      \left(
      D\difc.L.{u}{}
      \right)=
      0
    \end{math}
                &
    \begin{math}
      \difcp{u}{t}-
      \operatorname{div}
      \left(
      D\difc.L.{u}{}
      \right)=
      0
    \end{math}
    \\
    \hline
    Elíptica    &
    \begin{math}
      \operatorname{div}
      \left(
      vu-
      D\difc.L.{u}{}
      \right)=
      0
    \end{math}
                &
    \begin{math}
      -\operatorname{div}
      \left(
      D\difc.L.{u}{}
      \right)=
      0
    \end{math}
    \\
    \hline
    Hiperbólica &
    \begin{math}
      \difcp{u}{t}+
      \operatorname{div}
      \left(vu\right)=
      0
    \end{math}
                &
    \begin{math}
      \operatorname{div}
      \left(
      D\difc.L.{u}{}
      \right)=
      0
    \end{math}
  \end{tabular}
  \caption{Taxonomía de la ecuación.}
\end{table}

Este modelo base permite estudiar la cinética de sistemas de
reacciones químicas y el transporte de contaminantes en fenómenos
meteorológicos.

Nuestro estudio inicia con la presentación de la ley de conservación
escalar y la deducción de la ecuación de advección-difusión lineal.
Introducimos los esquemas de discretización de primer y segundo orden
para espacio y tiempo, así como los esquemas de alta resolución.
Para el caso bidimensional, utilizamos la técnica de separación
de la dimensión, en el cual resolvemos un sistema EDP unidimensional.
% ¿TODO: sistema hiperbólico?
Se estudia la consistencia, estabilidad y convergencia de los métodos
de primer orden Upwind (FOU), adelante en tiempo - centrado en
espacio (FTCS), Lax-Friedrichs, Leap-Frog, Lax-Wendroff.
.
Finalmente, se analiza los resultados de convergencia en las pruebas
numéricas sometidos a los esquemas presentados.

% TODO: El 1.5 de la tesis: outline se utiliza cuando se piensa escribir un libro.
% TODO: Si es necesario, mencionar los rangos de estabilidad de cierto esquema numerico
% TODO: Escribir el abstract el ingles porque son menos oraciones. porque por lo general
% en espanol se usan oraciones largas.
% marco conceptual sale del mapa conceptual (panorama)
% marco teorico: los teoremas y algoritmos
% Fase 3, Desarrollo del trabajo:  presentacion de los resultados tambien
% interrogante del problema: descripcion del problema.
% Conclusiones: Se ha llegado que la hipotesis supuesta se cumple. (enlace)